%
%	若手の会	規約
%
%
\documentstyle[titlepage]{jarticle}
%
\makeatletter
\def\labelenumi{第\theenumi 章 }
\def\theenumii{\arabic{enumii}}
\def\p@enumii{}
\def\labelenumii{第\theenumii 条 }
\def\theenumiii{\arabic{enumiii}}
\def\labelenumiii{\theenumiii }
\def\p@enumiii{}
\makeatother
%\def\labelenumi{第\theenumi 章 }
%\def\p@enumi{\theenumi}
%\def\theenumii{\arabic{enumii}}
%\def\p@enumii{enumi}
%\def\labelenumii{第\theenumii 条 }
%\def\theenumiii{\arabic{enumiii}}
%\def\labelenumiii{\theenumiii }
%
%\begin{titlepage}
\def\@maketitle{
 \@title
 \@author
 \@date
 \vspace*{5cm}
}
%\vspace*{5cm}
\title{\Huge 情報科学若手の会「規約」}
\author{\huge 情報科学若手の会}
\date{\LARGE 平成   8 年  8 月   1 日発行\\
                         平成 24 年  6 月   3 日改訂\\
                         平成 26 年  4 月 19 日改訂\\
                         平成 28 年 10 月  8 日改訂\\
					     平成 29 年  6 月 18 日改訂\\
					     平成 29 年 11 月 18 日改訂}
%\vspace*{5cm}
%\end{titlepage}
\begin{document}
\maketitle
%
\begin{enumerate}
%
%%%%%%%%%%%%%%%%%%%%%%%%%%%%%%%%%%%%%%%%%%%%%%%%%%%%%%%%%%%%%%%%%%%%%%%
\item 名称\\
%
\begin{enumerate}
\item (名称)本会は,情報科学若手の会と称する.\\
\end{enumerate}
%
%%%%%%%%%%%%%%%%%%%%%%%%%%%%%%%%%%%%%%%%%%%%%%%%%%%%%%%%%%%%%%%%%%%%%%%
\item	目的\\
%
\begin{enumerate}
\setcounter{enumii}{1}
\item \label{mokuteki} (目的) 本会は以下のことを目的とする.\\

	\begin{enumerate}
	\item	活発な討論の中から若手研究者ならではの斬新な発想を生み
出し,情報科学・情報工学の新しい可能性を考え,将来の夢を語り合う.\\
	\item	専門分野だけでなく情報科学のさまざまな分野で活躍する若
手どうしの討論を通して視野を広げる.\\
	\item	専門分野にこだわらず,情報科学の全般に渡る若手研究者の
横の繋がりを広げる.\\
	\end{enumerate}
%
\end{enumerate}
%
%%%%%%%%%%%%%%%%%%%%%%%%%%%%%%%%%%%%%%%%%%%%%%%%%%%%%%%%%%%%%%%%%%%%%%%
\item	組織\\
\begin{enumerate}
\setcounter{enumii}{2}
\item	(構成) 本会は幹事によって構成される.\\
%
\item	(幹事) 本会は次の幹事を置く.代表幹事(1 名),会計(1 名),一般幹事(若干 
名).但し,\\
\begin{enumerate}
	\item	幹事の任期は原則として 11 月 1 日より翌年 10 月 31 日迄
とする.\\
	\item	幹事の機能及び任免,並びに辞退,退会については別に之を
定める.\\
\end{enumerate}
%

\end{enumerate}
%%%%%%%%%%%%%%%%%%%%%%%%%%%%%%%%%%%%%%%%%%%%%%%%%%%%%%%%%%%%%%%%%%%%%%%
\newpage
\item	大会\\
%
\begin{enumerate}
\setcounter{enumii}{4}
\item	(主催) 本大会はプログラミング・シンポジウム委員会により主催さ
れる.\\
%
\item	(大会) 本会は年1回以上の大会を開く.やむを得ず開けない場合には
遅滞なくプログラミング・シンポジウム委員会に報告する.\\
%
\item (参加費) 本会は1回の大会において一定額の参加費を参加者
より徴収する.但し,幹事会が認めた者は之を減ぜられることがある.\\
%
\item (寄付・協賛) 本会は1回の大会において寄付・協賛を募ることができる.但し,
寄付・協賛の申請に対し幹事会において審議し,之を承認されなければな
らない.また,寄付・協賛の要項は別途幹事会にて策定することとする.\\
%
\end{enumerate}

%%%%%%%%%%%%%%%%%%%%%%%%%%%%%%%%%%%%%%%%%%%%%%%%%%%%%%%%%%%%%%%%%%%%%%%
\item	運営\\
%
\begin{enumerate}
\setcounter{enumii}{8}
\item	(活動) 通常の活動は代表幹事が之を統括する.\\
%
\item	(資金) 本会は大会における参加費,寄付, 協賛及びその他の収入をもって
運営資金とする.\\
%
\item (予算) 会計が幹事との合議の上,作成した予算案を幹事会が承認した時,予算は成立する.\\
%
\item (寄付金・協賛金) 
\begin{enumerate}
	\item	寄付・協賛は1口50,000円とする. \\
	\item	寄付・協賛は最低1口,最高5口とする. \\
\end{enumerate}
%
\item (決算) 決算は前年度の会計の決算報告を幹事会が承認する
ことにより成立する.\\
%
\item (資金運用) 本会の資金は,第 \ref{mokuteki} 条の目的以外に使用できな
い.また,その運用に関しては会計が一切の責任を持たねばならない.\\
%
\end{enumerate}
%
%%%%%%%%%%%%%%%%%%%%%%%%%%%%%%%%%%%%%%%%%%%%%%%%%%%%%%%%%%%%%%%%%%%%%%%%
%\item	大会における参加者\\
%%
%\begin{enumerate}
%\item	(参加者) 本規約に於いては年1回以上開催される大会への出席者を参加者とみなす.\\
%%
%\item	(参加資格) 大会への参加者は情報科学および関連分野で研究・実務に
%携わっている若手研究者・技術者であることを参加資格とする.\\
%%
%\item	(参加者の義務) 大会参加者は定められた参加費を一定期間内に納入す
%る義務を持つ.\\
%\end{enumerate}
%%
%%%%%%%%%%%%%%%%%%%%%%%%%%%%%%%%%%%%%%%%%%%%%%%%%%%%%%%%%%%%%%%%%%%%%%%
%\item	運営委員の参加及び退会\\
%%
%\begin{enumerate}
%\setcounter{enumii}{14}
%\item (資格) 運営委員は情報科学および関連分野で研究・実務に携わってい
%る若手研究者・技術者であることを参加資格とする.\label{shikaku}\\
%%
%\item (参加) 第 \ref{shikaku} 条の資格を有し,第 \ref{mokuteki} 条の目
%的に賛同し,幹事長により,参加を認められたものは参加をすることができる.\\
%%
%\item	(退会)
%\begin{enumerate}
%	\item	退会は原則として,運営委員の自由意志による.但し,幹事
%長から退会を勧告されたものは退会せねばならない.\\
%%	\item	退会しようとする者は書面をもって幹事長に届けねばならない.\\
%	\item	幹事の退会は原則として,幹事会の承認を必要とする.\\
%\end{enumerate}
%%
%\end{enumerate}
%%
%%%%%%%%%%%%%%%%%%%%%%%%%%%%%%%%%%%%%%%%%%%%%%%%%%%%%%%%%%%%%%%%%%%%%%%
\item	幹事の機能\\
%
\begin{enumerate}
\setcounter{enumii}{14}
\item	\label{daihyo}(代表幹事)
\begin{enumerate}
	\item	代表幹事は本会を統括し,対外的に本会を代表し,かつ大会責
任者を兼務し諸活動に関する一切の責任を持つ.\\
%	\item	代表幹事は本会に関して一切の決定権を有し,その内容及び結
%果に関して運営委員に対して責任を持たねばならない.\\
	\item	\label{kankoku}代表幹事は以下の条項に該当する幹事を,
幹事会の承認を得て退会を勧告することができる.\\
		\begin{enumerate}
			\item	本会の名誉を著しく毀損した者\\
			\item	本会の活動に著しく支障をきたすもの\\
		\end{enumerate}
	\item	代表幹事は第 \ref{daihyo} 条 \ref{kankoku} 項に基づいて退会勧告を行った幹事について,
幹事会の承認を得て退会させることができる.\\

%	\item 代表幹事は第 \ref{shikaku} 条の資格を有し,第 
%\ref{mokuteki} 条の目的に賛同する者を他の幹事と協議し,その承認を得て,
%参加させることができる.但し代表幹事は参加する運営委員を何らかの方法によ
%り,運営委員に紹介せねばならない.\\
%	\item	代表幹事は必要を認めた場合,幹事会を招集せねばならない.\\
	\item	代表幹事が職務を行う上での支障があると判断した場合は,
幹事会を招集し代表幹事の代行を立てることができる. \\
\end{enumerate}
%
\item	(会計)
\begin{enumerate}
	\item	会計は代表幹事と合議の上,予算案を作成し,
				幹事会に提出し,承認を求めなければならない.\\
	\item	会計は当該会計年度が終了した時は,速やかに決算書を作成
				し,翌年度の幹事会に報告し,承認を求め
なければならない.\\
\end{enumerate}
%
\end{enumerate}
%
%%%%%%%%%%%%%%%%%%%%%%%%%%%%%%%%%%%%%%%%%%%%%%%%%%%%%%%%%%%%%%%%%%%%%%%
\newpage
\item	幹事の任免及び辞任\\
%
\begin{enumerate}
\setcounter{enumii}{16}
\item	(幹事の選出) 幹事は幹事会に於いて選出されることを原則
				とする.\\
%
\item (幹事の補充) 幹事に欠員が生じた場合,代表幹事は速やかに幹事会を招
集し,幹事を補充せねばならない.但し,代表幹事に欠員が生じた場合または
代表幹事が職務を行うことが難しい場合は他の幹事が幹事会を招集せねばならない.\\
\end{enumerate}
%
%%%%%%%%%%%%%%%%%%%%%%%%%%%%%%%%%%%%%%%%%%%%%%%%%%%%%%%%%%%%%%%%%%%%%%%
\item    幹事会\\
%
\begin{enumerate}
\setcounter{enumii}{18}
\item    (幹事会) 幹事会は本会の最高決議機関であり,幹事選出の
                条件を審議決定する.\\
%
\item	代表幹事は必要と認めた場合には幹事会を招集できる.\\
\item	代表幹事は他の幹事の\(\displaystyle{\frac{1}{2}}\)以上の要請があった場合には幹事会を招集しなければならない.\\
\item	代表幹事が欠けている場合、幹事は全幹事の\(\displaystyle{\frac{1}{4}}\)以上の賛同を得て、幹事会を招集することができる.\\
\item    (成立) 
\begin{enumerate}
    \item 幹事会は幹事総数の~\(\displaystyle{\frac{3}{5}}\)~の出席
をもって成立する.\\
    \item 電子メールやインスタントメッセージングサービス等のオンラインによる幹事会の開催時は,審議事項に関して意見を述べることにより,出席したものと見なす.
\end{enumerate}
%
\item    (議決) 幹事会における議決は出席数の過半数をもって成立
                する.\\
%
%\item	(委任状)\\
%\begin{enumerate}
%	\item	白紙委任状は,代表幹事にその権限を委任したものとみなす.\\
%	\item	幹事選出,リコール,規約改正の場合は,白紙委任状を之を
%				認めない.\\
%\end{enumerate}
%
\end{enumerate}
%
%%%%%%%%%%%%%%%%%%%%%%%%%%%%%%%%%%%%%%%%%%%%%%%%%%%%%%%%%%%%%%%%%%%%%%%
\item	附則\\
%
\begin{enumerate}
\setcounter{enumii}{21}
\item	本規約は平成 29 年  6 月 18 日から施行する. \\
%
%\item	(寄付金)
%\begin{enumerate}
%	\item	寄付は 1 口 50,000 円とする.\\
%	\item	寄付は最低 1 口,最高 6 口とする.
%\end{enumerate}
%
\end{enumerate}
%
\end{enumerate}
%
\end{document}
